\documentclass{article}
\usepackage{graphicx} % Required for inserting images
\usepackage{longtable}

\newcommand{\ben}{\begin{enumerate}}
\newcommand{\een}{\end{enumerate}}
\newcommand{\bei}{\begin{itemize}}
\newcommand{\eei}{\end{itemize}}
\newcommand{\ie}{\item}

\title{A review of the use of Artificial Intelligence in Ecology}
\author{Sergio Poo Hernandez}
\date{September 2025}

\begin{document}

\maketitle

\section{Introduction}

Artificial Intelligence (AI) methods have been used to automate tasks since the concept was created. As new and more advanced algorithms are developed the number of tasks that are automate has increased. This is no different in the field of Ecology. But how has the field adapted to new technologies? How eager are ecologists to adopt new methods and abandon older ones? In this review we aim to get a better understanding of the adoption trends in the field of Ecology. The following section highlights the questions we are interested in answering.

\section{Questions}
\ben
\item What ecological applications of Artificial Intelligence (AI) have been reviewed, and which ecological study types and AI model types (e.g., supervised, deep learning) are most frequently considered in this secondary literature?
\item What are the prevailing research trends, identified challenges, and future directions in existing reviews on the use of AI in ecology? 
\ben
\item Are new AI methods widely adopted in ecological applications as soon as they are available, or is there a delay in adoption?
\item Are older methods still used, and in what areas?
\een
\item Based on a synthesis of the secondary literature, what are the most significant knowledge gaps and methodological limitations in the application of AI to solve ecological problems?
\item What is the reach of these studies in policy generation and in their respective fields?
\ben
\item How often are the reviews used by others?
\item What are the collaboration patterns in the existing systematic reviews on this topic?
\een
\een

\section{Inclusion and exclusion criteria}
\subsection{Population}
Inclusion: reviews must focus on AI (or Machine Learning and Deep Learning) applications in any biological system, including organismal, or ecological levels. Studies on mixed domains are included if biological applications are a substantial focus.

Exclusion: Exclude reviews where the analytical focus is exclusively on non-biological systems (e.g., finance, engineering, computer vision without a biological application) or on humans for purely clinical/biomedical purposes, or application in biological systems on the molecular or cellular levels.

\subsection{Exposure}
Inclusion: Reviews must explicitly analyze Machine Learning (ML), Deep Learning (DL), or Artificial Intelligence (AI) as a distinct set of tools. This includes supervised, unsupervised, and reinforcement learning methods.

Exclusion: Exclude reviews focusing solely on traditional statistical methods (e.g., linear regression, ANOVA) without framing them in the context of AI. Exclude reviews where AI is not a central theme or cannot be distinguished from other computational methods. Exclude reviews focusing only on data collection for remote sensing.

\subsection{Comparator}
Not applicable

\subsection{Outcome}
Inclusion: The review must discuss the outcomes of applying AI, such as prediction, classification, clustering, or generation of biological data. This includes outcomes such as modeling ecological systems or classifying species from images.

Exclusion: Exclude studies focused only on the theoretical or mathematical aspects of AI algorithms without discussing their application or impact in a biological context. Exclude papers on software development that do not review the outcomes of its use.

\subsection{Study type}
Inclusion: Only secondary literature, specifically systematic reviews, systematic maps and meta-analyses that synthesize primary literature, will be included. Studies must be peer-reviewed.

Exclusion: Exclude primary research articles, editorials, opinion pieces, book chapters, and narrative/scoping reviews.

\section{Benchmark papers}
\bei
\item Algorithms going wild – A review of machine learning techniques for terrestrial ecology
\item Towards the fully automated monitoring of ecological communities
\item Unlocking the potential of deep learning for marine ecology: overview, applications, and outlook
\item Application of machine-learning methods in forest ecology: recent progress and future challenges
\item Deep learning as a tool for ecology and evolution
\item Machine learning and deep learning—A review for ecologists
\item MACHINE LEARNING METHODS WITHOUT TEARS: A PRIMER FOR ECOLOGISTS
\item ‘Small Data’ for big insights in ecology
\item Machine learning in landscape ecological analysis: a review of recent approaches
\item Artificial Intelligence in Landscape Ecology: Recent Advances, Perspectives, and Opportunities
\item Generative AI as a tool to accelerate the field of ecology
\eei

\section{Search strings}
\subsection{Web of Science}
1st search string

TS=(("Artificial intelligence" OR "Machine learning" OR "Deep learning" OR "neural network" OR "Reinforcement learning" OR “Large Language Models”) AND (Ecolog* OR (Evolution* NEAR/3 biology) NOT medic* NOT microb* NOT land*) AND (("review" NOT (review NEAR/3 human)) OR "systematic map" OR "survey" OR "recent advancements" OR insight)) AND SU=("Ecology" OR "Biology" OR " Evolutionary biology" OR "Biodiversity" OR "Conservation" OR "Life Sciences" NOT "Microbiology" NOT "Physical Sciences")

Results in $1184$ articles.

2nd search string

TS=(("Artificial intelligence" OR "Machine learning" OR "Deep learning" OR "neural network" OR "Reinforcement learning" OR “Large Language Models”) AND (review* OR "systematic map*" OR "survey*" OR "recent advancement*" OR insight*)) AND SU=("Ecology" OR " Evolutionary biology" OR "Biodiversity" OR "Conservation")

Putting limitations for Web of Science Categories:

Include:
\bei
\item Remote Sensing
\item Ecology
\item Marine Freshwater Biology
\item Evolutionary Biology
\item Plant Sciences
\item Biology
\item Zoology
\item Entomology
\item Ornithology
\item Fisheries
\eei
NOT include:
\bei
\item Environmental Sciences
\eei

Results in $992$ articles.

\subsection{Scopus}
SUBJAREA ( AGRI ) TITLE-ABS ( ( "Artificial intelligence" OR "Machine learning" OR "Deep learning" OR "neural network" OR "Reinforcement learning" OR "Large Language Models" ) AND ( Ecolog* OR Evolution* ) AND ( "review" OR "systematic map" OR "survey" OR "recent advancements" OR insight ) )

Results in $1354$ articles.

\section{Code book}
%\begin{table}[h]
    \begin{longtable}{p{1.5cm}p{3cm}p{3cm}p{3cm}}
%    \centering
        \hline
         \textbf{Variable} & \textbf{Description} & \textbf{Format/Allowed values} & \textbf{Notes}\\
         \hline
        extractor initial & Initials of the data extractor & Text (e.g., ALM, ML) & For inter-rate calibration and audit trail\\
        study id & Unique ID assigned to the review. & Text (e.g., pichler\_2023) & Use firstauthor\_year format\\
        doi & Digital Object Identifier of the review & Text (e.g., 10.1111/2041-210X.14096) & Extract from journal metadata\\
        year & Year of publication & YYYY & Extract from journal metadata\\
        review type & The main type of secondary review & Systematic review, Systematic map, Meta-analysis, Narrative review, Scoping review & Assign one value based on what the authors explicitly state\\
        primary study design & The main type of study design of primary studies included in the review & Experimental, Observational, Simulation, Mixed, Unclear & Use "Mixed" if multiple apply. Use "Unclear" if not explicitly stated\\
        mentions n studies & Does the review report the number of studies it looked at? & Yes, No & Must be explicitly reported. If yes record n\_studies, if no leave n\_studies empty\\
        n studies & The number of primary studies synthesized in the review & Integer & Must be explicitly reported. Refers to the number of publications, not experiments or effect sizes\\
        biology domain & The primary biological domain(s) of focus & Ecology, Bioinformatics, Medicine, Systems Biology, Mixed, Other, Unclear & Multiple categories are allowed. "Mixed" for reviews covering many domains\\
        data source type & The source of data used in the primary studies reviewed & Field-collected, Lab-generated, Public repository, Mixed, Unclear & Refers to the origin of the data that ML models are trained on\\
        ai model category & Broad category of AI models discussed & Supervised, Unsupervised, Reinforcement, Mixed & Multiple categories are allowed\\
        specific algorithms & Does the review focus on a specific, named algorithm or architecture? & Yes, No, Unclear & Yes for a focus like "Random Forest" or "Convolutional Neural Networks."\\
        specific algorithms comment & If yes, what specific algorithms or architectures are mentioned? & Free text & List the exact terms used by the authors (e.g., SVM, GAN, AlphaFold)\\
        ai algorithms & What specific AI algorithms are reviewed & CNN, RNN, LLM, Decision trees, boosted regression trees, random forest, SVM & Multiple categories, only some examples are included since more classifications may be acceptable\\
        ai algorithms comments & Any observations about the Ai algorithms listed in the review & Free text & \\
        compared to stats & Does the review explicitly compare AI models to traditional statistical methods? & Yes, No, Unclear & Yes if there is a direct comparison to methods like linear/logistic regression\\
        application goal & The broad category of the task or goal the ML models are used for & Classification, Prediction/Regression, Clustering, Dimensionality Reduction, Generation, Other, Mixed & Multiple categories are allowed. This describes what the model is doing\\
        application goal comment & A specific description of the application goals & Free text & Provide more detail, e.g., "Classifying species from images," "Predicting protein structure."\\
        datatype used & Categories of the data used for the AI models & Audio, Video, Images, Numeric, Mixed & Multiple categories are allowed\\
        performance moderators & Does the review assess factors that modify or influence ML model performance? & Yes, No, Unclear & Yes if the review discusses how things like dataset size, data quality, or hyperparameter tuning affect outcomes\\
        performance moderators comment & If yes, name the interacting factors or moderators considered & Free text & List exact terms used (e.g., sample size, feature engineering, class imbalance)\\
        temporal analysis & Does the review do any temporal analysis of the data reviewed? & Yes, No & Yes if there is a comparison through time of any of the reviewed data. No otherwise\\
        temporal analysis comment & If yes, what kind of analysis was made & Free text & List what elements are compared through time\\
        identified gaps & Does the review explicitly identify gaps in the research? & Yes, No, Unclear & Yes if the authors state what is missing or needs more work\\
        identified gaps comment & If yes, what research gaps are mentioned? & Free text & List the exact gaps identified (e.g., Need for more interpretable models, Lack of causal inference methods)\\
        general notes & Any other relevant comments or observations & Free text & Use for important context not captured elsewhere (e.g., "Main focus was on deep learning, but briefly mentioned classical ML")\\
         \hline
    \end{longtable}
    %\label{tab:satMapsvscoordinates}
%\end{table}

\section{Bibliometric data}
\begin{longtable}{p{1.5cm}p{3cm}p{3cm}p{3cm}}
%    \centering
        \hline
         \textbf{Variable} & \textbf{Description} & \textbf{Format/Allowed values} & \textbf{Notes}\\
         \hline
         doi & Digital Object Identifier of the review & Text (e.g., 10.1111/2041-210X.14096) & Extract from journal metadata\\
         authors & Full list of authors & Free text & \\
         affiliation & Institutional and country affiliations & Free text & \\
         corr country affiliation & Country of first or corresponding author & Country name & \\
         n authors & Number of authors & Integer & \\
         citations scopus & Citation counts from Scopus & Integer & \\
         citations web of science & Citation counts from Web of Science & Integer & \\
         policy citations & Policy citation counts & Integer & \\

\end{longtable}

\section{Time estimations}

We estimate the time for data extraction for each type of review as the average time it took to do the extraction on the papers that are in our benchmark. The average is $14.5$ minutes. We expect to perform data extraction on 60 reviews, so data extraction is estimated to take $870$ minutes.

\section{Pilot}


\end{document}
