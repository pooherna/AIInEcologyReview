\documentclass{article}
\usepackage{graphicx} % Required for inserting images

\title{AIInEcologyReviewProtocol}
\author{pooherna }
\date{September 2025}

\begin{document}

\maketitle

\section{Questions}
What ecological applications of Artificial Intelligence (AI) have been reviewed, and which ecological study types and AI model types (e.g., supervised, deep learning) are most frequently considered in this secondary literature?
How many of the reviews have been systematic and what are the scope of them?
What are the prevailing research trends, identified challenges, and highlighted future directions in the existing reviews on the use of AI in ecology? 
Are new AI methods widely adopted in ecological applications as soon as they are available, or is there a delay in adoption?
Are there older methods abandoned or still useful in some areas?
Based on a synthesis of the secondary literature, what are the most significant knowledge gaps and methodological limitations in the application of AI to solve ecological problems?
What is the reach of these studies in policy generation and in their respective fields?
How often are the reviews used by others?
Is there a difference between systematic and narrative reviews?
Origin of the studies of interest.

\section{Inclusion and exclusion criteria}
\subsection{Population}
Inclusion: reviews must focus on AI (or Machine Learning and Deep Learning) applications in any biological system, including organismal, or ecological levels. Studies on mixed domains are included if biological applications are a substantial focus. Domain is in journal
Exclusion: Exclude reviews where the analytical focus is exclusively on non-biological systems (e.g., finance, engineering, computer vision without a biological application) or on humans for purely clinical/biomedical purposes, or application in biological systems on the molecular or cellular levels.

\subsection{Exposure}
Inclusion: Reviews must explicitly analyze Machine Learning (ML), Deep Learning (DL), or Artificial Intelligence (AI) as a distinct set of tools. This includes supervised, unsupervised, and reinforcement learning methods.
Exclusion: Exclude reviews focusing solely on traditional statistical methods (e.g., linear regression, ANOVA) without framing them in the context of AI. Exclude reviews where AI is not a central theme or cannot be distinguished from other computational methods. Exclude reviews focusing only on data collection for remote sensing.

\subsection{Comparator}
Not applicable

\subsection{Outcome}
Inclusion: The review must discuss the outcomes of applying AI, such as prediction, classification, clustering, or generation of biological data. This includes outcomes like modeling ecological systems or classifying species from images.
Exclusion: Exclude studies focused only on the theoretical or mathematical aspects of AI algorithms without discussing their application or impact in a biological context. Exclude papers on software development that do not review the outcomes of its use

\subsection{Study type}
Inclusion: Only secondary literature, specifically systematic reviews, systematic maps, meta-analyses, and narrative/scoping reviews that synthesize primary literature, will be included. Studies must be peer-reviewed.
Exclusion: Exclude primary research articles, editorials, opinion pieces, book chapters.

\section{Benchmark papers}
Algorithms going wild – A review of machine learning techniques for terrestrial ecology
Towards the fully automated monitoring of ecological communities
Unlocking the potential of deep learning for marine ecology: overview, applications, and outlook
Application of machine-learning methods in forest ecology: recent progress and future challenges
Deep learning as a tool for ecology and evolution
Machine learning and deep learning—A review for ecologists
MACHINE LEARNING METHODS WITHOUT TEARS: A PRIMER FOR ECOLOGISTS
‘Small Data’ for big insights in ecology
Machine learning in landscape ecological analysis: a review of recent approaches
Artificial Intelligence in Landscape Ecology: Recent Advances, Perspectives, and Opportunities
Generative AI as a tool to accelerate the field of ecology

\section{Search strings}
\subsection{Web of Science}
1st search string
TS=(("Artificial intelligence" OR "Machine learning" OR "Deep learning" OR "neural network" OR "Reinforcement learning" OR “Large Language Models”) AND (Ecolog* OR (Evolution* NEAR/3 biology) NOT medic* NOT microb* NOT land*) AND (("review" NOT (review NEAR/3 human)) OR "systematic map" OR "survey" OR "recent advancements" OR insight)) AND SU=("Ecology" OR "Biology" OR " Evolutionary biology" OR "Biodiversity" OR "Conservation" OR "Life Sciences" NOT "Microbiology" NOT "Physical Sciences")
2nd search string
TS=(("Artificial intelligence" OR "Machine learning" OR "Deep learning" OR "neural network" OR "Reinforcement learning" OR “Large Language Models”) AND (review* OR "systematic map*" OR "survey*" OR "recent advancement*" OR insight*)) AND SU=("Ecology" OR " Evolutionary biology" OR "Biodiversity" OR "Conservation")
Putting limitations for Web of Science Categories:
Include:
Remote Sensing
Ecology
Marine Freshwater Biology
Evolutionary Biology
Plant Sciences
Biology
Zoology
Entomology
Ornithology
Fisheries
NOT include:
Environmental Sciences


\end{document}
